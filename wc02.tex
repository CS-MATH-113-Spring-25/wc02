\documentclass[a4paper]{exam}

\usepackage{amsmath,amssymb, amsthm}
\usepackage{geometry}
\usepackage{graphicx}
\usepackage{hyperref}
\usepackage{titling}
\usepackage{forest}




% Header and footer.
\pagestyle{headandfoot}
\runningheadrule
\runningfootrule
\runningheader{CS/MATH 113, SPRING 2025}{WC 02: Propositional logic}{\theauthor}
\runningfooter{}{Page \thepage\ of \numpages}{}
\firstpageheader{}{}{}

% \printanswers %Uncomment this line

\title{Weekly Challenge 02: \LaTeX\; Propositional logic}
\author{Blingblong} % <=== replace with your student ID, e.g. xy012345
\date{CS/MATH 113 Discrete Mathematics\\Habib University\\Spring 2025}

\qformat{{\large\bf \thequestion. \thequestiontitle}\hfill}
\boxedpoints

\begin{document}
\maketitle

\begin{questions}
  
\titledquestion{Prank gone too far} 
Skips had warned mode Mordecai and Rigby to not prank Muscle man, but they didn't listen.
Their plan was to first speak a made-up language that Muscle man won't understand, in this language they had two words ``gazorpazorp'' and ``plumbus'', one meant ``yes'' and the other meant ``no'', muscle man has no idea which is which. They keep prank calling Muscle man in that made-up language.
The prank went on so long that it gets noticed by the Master Prank Caller, who is not happy with it. He along with the wizard cursed Mordecai and Rigby to either always speak the truth or always tell a lie. But as they speak the made-up language, no one has any idea is they are always telling the truth or always lying. Muscle man has only one way to remove the curse, it is to figure out if they always tell the truth or always lie. But here's the catch, Muscle man can only ask them one question. 

What one question can Muscle man ask them to figure out if they are speaking the truth or lying? Justify your answer.
\begin{solution}
    % Add your solution here
\end{solution}

\titledquestion{How about them apples?}
  \begin{minipage}{.3\linewidth}
  \centerline{\includegraphics[width=\textwidth]{picard}}
\end{minipage}
\begin{minipage}{.65\linewidth}
  The \href{https://en.wikipedia.org/wiki/Replicator_(Star_Trek)}{replicator} aboard USS Enterprise has developed a fault---synthesized apples have insufficient nutrition but are otherwise identical to regular apples. Doctor \href{https://memory-alpha.fandom.com/wiki/Beverly_Crusher}{Beverly Crusher} is on the case. Scanning a bunch of apples, her \href{https://en.wikipedia.org/wiki/Medical_tricorder}{tricorder} can indicate if the bunch contains any faulty apples, but it cannot identify them.
\end{minipage}

  Dr. Crusher is investigating a bunch of 5 apples out of which, 1 is known to be faulty. Describe how she can identify the faulty apple in no more than 3 tricorder scans. Furthermore, what is the minimum number of scans that Dr. Crusher needs to perform in order to guarantee finding the single faulty apple in a bunch of size $n$? Justify your answer.

  \begin{solution}
    % Add your solution here
    \end{solution}


      
\end{questions}
\end{document}

%%% Local Variables:
%%% mode: latex
%%% TeX-master: t
%%% End:
